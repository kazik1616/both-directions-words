\documentclass[a4paper,11pt]{article}
\usepackage{latexsym}
\usepackage[polish]{babel}
\usepackage[utf8]{inputenc}
\usepackage[MeX]{polski}


\author{Krzysztof Kaźmierczyk}
\title{Simple text processing \\ 
\large{Detect and write down all the words that can be read in both directions}} 
 

\begin{document}

\maketitle 

\section{Realizacja projektu}
Projekt został zrealizowany w języku \texttt{Python} ze względu na dosyć dobre wsparcie kodowania utf-8. Do zebrania słów występujących w języku polskim użyto następujących tekstów:
\begin{itemize}
\item B. Prus - Faraon
\item B. Prus - Lalka
\item H. Sienkiewicz - Potop
\item h. Sienkiewicz - Quo Vadis
\item S. Żeromski - Popioły
\item S. Żeromski - Przedwiośnie
\end{itemize}


\section{Wyniki}
W tekście znaleziono 110225 różnych słów. Wśród tych słów znaleziono 357 słów, które można czytać w dwie strony, z czego 94 słowa są palindromami.

\end{document}  
